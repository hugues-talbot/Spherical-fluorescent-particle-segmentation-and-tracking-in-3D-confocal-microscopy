  
  % ----------------------
  % T O P
  \vspace*{-0.2em}

  \begin{exampleblock}{\bf At a glance}
    \centering

    \vspace*{.1em}

  \begin{minipage}{.154\linewidth}
    \includegraphics[width=\linewidth]{IMG_0336}
   \captionof{subfigure}{Input.}
    \label{fig:glob.input}
  \end{minipage}
  ~
  \begin{minipage}{.154\linewidth}
    \includegraphics[width=\linewidth]{IMG_0336_temp_lab_filtered}
    \captionof{subfigure}{\Lab filtered.}
    \label{fig:glob.labfiltered}
  \end{minipage}
  ~
  \begin{minipage}{.154\linewidth}
    \includegraphics[width=\linewidth]{IMG_0336_temp_grad}
    \captionof{subfigure}{Gradient.}
    \label{fig:glob.gradient}
  \end{minipage}
  ~
   \begin{minipage}{.154\linewidth}
    \includegraphics[width=\linewidth]{IMG_0336_temp_wst_basins}
    \captionof{subfigure}{Basins.}
    \label{fig:glob.basins}
  \end{minipage}
  ~
   \begin{minipage}{.154\linewidth}
    \includegraphics[width=\linewidth]{IMG_0336_temp_chunks_selection_alt}
    \captionof{subfigure}{Chunks.}
    \label{fig:glob.chunks}
  \end{minipage}
  ~
   \begin{minipage}{.154\linewidth}
    \includegraphics[width=\linewidth]{IMG_0336_out}
    \captionof{subfigure}{Decision.}
    \label{fig:glob.decision}
  \end{minipage}

     
    \vspace*{-.8em}
    \begin{columns}[t]
      \hspace*{3em}\begin{column}{0.45\textwidth}
%        \centering
%        \begin{minipage}[t]{0.98\linewidth}
          \begin{description}
          \item[{\bf Problem:}] ~ %
            \begin{itemize}
            \item Real time document detection in smartphone videos is challenging
            \end{itemize} \bigskip  % \bigskip\bigskip
            %
          \item[{\bf Why our approach is intersting:}] ~ %
            \begin{itemize}
            \item no apriori on documents on images
            \item it is light enouth to be run on smartphones
            \end{itemize}
            %
          \end{description}
%        \end{minipage}
      \end{column}
      ~
      \begin{column}{0.43\textwidth}
%        \centering
%        \begin{minipage}[t]{1.0\linewidth}
%          \vspace*{1ex}
          \begin{description}
          \item[{\bf Conclusion:} ] ~ %
          Our method is:
            \begin{itemize}
          \item robust to noise
          \item fast: 0.04 s per frame
          \item light 
            \end{itemize}
          Jaccard coefficient of 0.9 in average on SmartDoc 2015

          \end{description}
%        \end{minipage}
      \end{column}
    \end{columns}
  \end{exampleblock}

%%   % ###################################################################
%%   % ###################################################################
%%   % ###################################################################
  \vspace*{0.4em}

  % ###################################################################
  % ###################################################################
  % ###################################################################

  \vspace*{-0.8em}

  \begin{columns}[t]
    %
    \begin{column}{0.49\textwidth}
      \begin{block}{\bf Description of the method}
      \begin{description}
        \item[{\bf Four main steps:}] ~
        \begin{itemize}
        \item  Reduce size and change of color space:  \\%
            $\leadsto$ each frame is reduced (e.g. from 280 $\times$ 720 px to 180 $\times$ 100 px),\\
            $\leadsto$ then converted to \Lab space,\\
            $\leadsto$ and then regularized: morphological closing on \ensuremath{L} and erosion on \ensuremath{a}. 
	   \item  Segment the image into regions: \\ %
	        $\leadsto$ morphological thick gradient is computed on each component of \Lab and summed up,\\
	        $\leadsto$   this summed gradient is filtered (regional minima are removed), \\
	        $\leadsto$ and a watershed transform is applyied.
       \item  Extract line chunks from region contours:  \\%
            $\leadsto$ classical Hough transform is applyied on the binary watershed,\\
            $\leadsto$ lines obtained are cut into chunks, keeping only the closest to the watershed line, and some parameters (orientation etc.) are stored.\\
		  $\leadsto$ Redundant chunks are removed.
       \item  Find the document boundaries: \\
           $\leadsto$ classification of chunks (top, bottom, left, right) \\
           $\leadsto$ paris of compatible chunks are made (top-left for example)\\
           $\leadsto$ find the best path using energy criterion based on the distance to the watershed line
      \end{itemize}
        \end{description}
        %
        \begin{figure}[t]
  \centering
  \includegraphics[width=.24\linewidth, angle = 90]{IMG_0020_out}~
  \includegraphics[width=.24\linewidth, angle = 90]{IMG_0333_out}~
  \includegraphics[width=.24\linewidth, angle = 90]{IMG_0348_out}
%  \medskip
  
  \caption*{Some results. \label{fig:results}}
  \vspace*{-.7em}
\end{figure}

      \end{block}
    \end{column}
    %
    \begin{column}{0.49\textwidth}
      \begin{block}{\bf Results}
      \begin{description}
      
      \item[{\bf Quantitative results on SmartDoc 15 Challenge database}]
      
        \begin{table}[t]
                \ \\
        \vspace{1cm}
\centering
\renewcommand{\arraystretch}{1.2}\setlength{\tabcolsep}{3pt}
\resizebox{.99\linewidth}{!}{
\begin{tabular}{|c|cccc|c|}
    \hline
    {\vspace*{.1em}\bf Method} & {\bf set 01} & {\bf set 02} & {\bf set 03} & {\bf set 04} & {\bf ~runtime~}\\
    \hline\hline
        {Xu \etal}~\cite{xu.2017.pami} & 0.997 & 0.987 & 0.999 & 0.994 & \textgreater 1min\\ % global score = .9816
    {LRDE}~\cite{burie.2015.icdar} & 0.987 & 0.977 & 0.989 & 0.984 & \textgreater 1min\\
%    {ISPL-CVML} & 0.987 & 0.965 & 0.985 & 0.977 \\
    {Leal \etal~\cite{leal.2016.lacci} best} & 0.961 & 0.944 & 0.965 & 0.930 & 0.43s \\
    {\it SmartDoc avg.}~\cite{burie.2015.icdar} & \it 0.946 & \it 0.903 & \it 0.938 & \it 0.812 & \it ? \\
    {Leal \etal~\cite{leal.2016.lacci} fastest} & 0.921 & 0.849 & 0.909 & 0.840 & 0.10s \\
    \hline
    {Our} & 0.905 & 0.936 & 0.859 & 0.903 & {\bf 0.04s} \\
    \hline
\end{tabular} 
}
%\vspace*{-.2em}
\caption*{Quantitative results (Jaccard coefficient) of automatic document detection methods applyied on 4 datasets. ``SmartDoc avg.'' corresponds to the average of the methods of the challenge.\label{tab:quantitative}}
%\vspace*{-.9em}
\end{table}


    \item[{\bf Robustness of our method:}]
    %
    
    
        \begin{figure}[t]
        \ \\
        \vspace{1cm}
  \centering  %trim=left bottom right top, clip
  %
  \subcaptionbox{Low-light environment + noise.\vspace*{.7em} \label{fig:robust.lowlight}}{
  \includegraphics[trim=120 708 500 120, clip, width=.24\linewidth, angle = 90]{IMG_0336_dark}~
  \hfil
  \includegraphics[width=.24\linewidth, angle = 90]{IMG_0336_dark_wst}~
  \hfil
  \includegraphics[width=.24\linewidth, angle = 90]{IMG_0336_dark_out}}
  %
  \subcaptionbox{High-light environment + noise.\vspace*{.7em} \label{fig:robust.highlight}}{
  \includegraphics[trim=120 708 500 120, clip, width=.24\linewidth, angle = 90]{brighter}~
  \hfil
  \includegraphics[width=.24\linewidth, angle = 90]{brighter_chunks}~
  \hfil
  \includegraphics[width=.24\linewidth, angle = 90]{brighter_out}}
  %
  \subcaptionbox{Defocus + flare. \label{fig:robust.defocus}}{
  \includegraphics[trim=120 708 500 120, clip, width=.24\linewidth, angle = 90]{defocus2}~
  \hfil
  \includegraphics[width=.24\linewidth, angle = 90]{defocus2_chunks}~
  \hfil
  \includegraphics[width=.24\linewidth, angle = 90]{defocus2_out}}
  \vspace*{-.5em}
  %  
  \caption*{Examples of the robustness of our method. From left to right: detail of the input image, chunks, final decision.\label{fig:robust}}
  %\vspace*{-.7em}
\end{figure}

    \end{description}
      \end{block}
    \end{column}
    

  \end{columns}



%% %###################################################################################
%% %###################################################################################


  \vspace*{0.4em}

      
  \begin{block}{\bf Selected bibliography}

    \begin{thebibliography}{10}
      \newcommand{\tinysep}{\vspace*{-.1em}}

      \renewcommand{\baselinestretch}{.98}
      \relsize{-0.5}
      
      \tinysep

%    \bibitem{icip}  {\color{MyGray}{Y.~Xu, T.~G\'eraud, and I.~Bloch,
%        ``{\bf From neonatal to adult brain {MR} image segmentation in
%          a few seconds using {3D}-like fully convolutional network
%          and transfer learning},'' in {\it ICIP}, pp. 4417--4421, 2017.}
%      {\color{purple}\bf~~$\leadsto$~~\url{http://publications.lrde.epita.fr/xu.17.icip}}}
%      \tinysep
%
%      \bibitem{brainles} {\color{MyGray}{Y.~Xu, \'E.~Puybareau,
%          T.~G\'eraud, and J.~Chazalon, ``{\bf White matter
%            hyperintensities segmentation in a few seconds using fully
%            convolutional network and transfer learning},'' in {\it
%            BrainLes (MICCAI 2017 competition)}, pp. 501--514,
%          vol. 10670 of LNCS, Springer, 2018.}}  \tinysep
%
%    \bibitem{vgg}
%      {\color{MyGray}{K.~Simonyan and A.~Zisserman,
%      ``{\bf Very deep convolutional networks for large-scale image recognition},''
%      {\it CoRR}, vol. abs/1409.1556, 2014.}}
%      \tinysep
%
%    \bibitem{adubrain}
%      {\color{MyGray}{A.~M. Mendrik {\it et~al.},
%      ``{\bf {MRBrainS} challenge: Online evaluation
%      framework for brain image segmentation in {3T MRI} scans},''
%          {\it Computational Intelligence and Neuroscience}, 2015.}
%        {\color{purple}\bf~~$\leadsto$~~\url{http://mrbrains13.isi.uu.nl/results.php}}}
%      \tinysep
%
%    \bibitem{transfer}
%      {\color{MyGray}{J.~Long, E.~Shelhamer, and T.~Darrell,
%      ``{\bf Fully convolutional networks for semantic segmentation},''
%      in {\it CVPR}, pp. 3431-3440, 2015.}}
%      \tinysep
%
%    \bibitem{patch3D}
%      {\color{MyGray}{H.~Chen {\it et al.},
%      ``{\bf VoxResNet : Deep voxelwise residual networks for volumetric brain segmentation},''
%      \url{https ://arxiv.org/abs/1608.05895}, 2016.}}
%      \tinysep
%      
%    \bibitem{patch2Dhalf}
%      {\color{MyGray}{K.~Fritscher {\it et al.},
%      ``{\bf Deep neural networks for fast segmentation of 3D medical images},''
%      in {\it MICCAI}, vol. 2, pp. 158-165, 2016.}}
%      \tinysep
      
    \end{thebibliography}
    
  \end{block}

